\documentclass[]{article}

\usepackage{graphicx} %for å inkludere grafikk
\usepackage{verbatim} %for å inkludere filer med tegn LaTeX ikke liker
\usepackage{tabularx}
\usepackage{booktabs}
\usepackage{amsmath}
\usepackage{float}
\usepackage{color}
\usepackage{listings}
\usepackage{physics}
\usepackage{hyperref}
\usepackage{subfig}
\usepackage{mhchem}
\usepackage{natbib}
%opening
\title{}
\author{}

\begin{document}
	
\title{Fission: a TALYS report}
\author{Dorthea Gjestvang }
\date{December 2017}

\maketitle

\begin{abstract}

\end{abstract}

\section{Introduction}

\section{Theory}
\subsection{Fission barrier}
When observing the average binding energy per nucleon, as seen in Figure \ref{fig:binding_energy_per_nucleon}, one can see that for $A \approx 60$ the binding energy per A peaks, and then starts to decrease for increasing A. For nuclei situated to the right of this peak, it is thus possible to release energy by splitting in two lighter fragments, that is to fission. However, when studying the chart of nuclei , one will observe that only a handfull of nuclei have spontaneous fission as their main decay mode, and they are generally heavy elements far from the valley of stability. This  is due to the so-called fission barrier. 
\par
\vspace{3mm}

\begin{figure}
	\centering
	\includegraphics[scale=0.6]{binding_energy_per_nucleon.png}
	\caption{Average binding energy per nucleon. Figure from K.S. Krane p.67 \cite{Krane1988}}
	\label{fig:binding_energy_per_nucleon}
\end{figure}


 \noindent In the liquid model, the fission barrier is described as a smooth, parabolic barrier, and is shown in Figure \ref{fig:smooth_fission_barrier}. It describes the energy needed to separate the two fission fragments as a function of separation distance. The fission barrier is due to both the nuclear force and the Coloumb force. A picture on why there is a fission barrier, is that in order for the two would-be fission fragments to separate, they have to get out of the potential well that is the nuclear force, and then pass the Coloumb barrier surrounding the nucleus. Thus energy must be applied for the fragments to be able to separate. This energy needed is often referred to as the activation energy, and is shown in Figure \ref{fig:smooth_fission_barrier} as the difference in energy between the ground state of the nucleus, and the maximum of the fission barrier potential. 
 
 \par
 \vspace{3mm}
 
 \begin{figure}
 	\centering
 	\includegraphics[scale=0.7]{smooth_fission_barrier.png}
 	\caption{The smooth fission barrier, illustrating the activation energy. Figure from K.S. Krane p.481 \cite{Krane1988}}
 	\label{fig:smooth_fission_barrier}
 \end{figure}

\noindent The energy released in fission is about the same as the activation energy needed to overcome the fission barrier. Those nuceli that have an energy release though fission that allows them to overcome most of the fission barrier, have a highter probability of tunnelling through the barrier \cite{Krane1988} (p.481). This is a known effect from quantum physics: the thinner the potential barrier, the larger the probability of tunnelling. These nuclei that see a thin potential barrier are thus called the spontaneously fissioning nuclei. However, most of the nuceli are not able to overcome much fission barrier by themselves, and thus they see a thick potential, and the probability of tunnelling is vanishing. They are therefore not unstable to spontaneous fission. This explains why the number of spontaneously fissioning nuclei are so few, even though it is energetically possible for plenty of the heavier nuclei to fission.

\par
\vspace{3mm}

\noindent This far I´ve only described the fission barrier using the liquid drop model. However, the liquid drop model is not a complete description of the nuclei, and shell effects have a impact when studying the probability to fission. When including the effects of the single particle shell structure, the fission barrier is changed to a barrier with two humps, called the double-humped fission barrier \cite{Krane1988} (p.495). The double-humped fission barrier is shown in Figure \ref{fig:double_humped_fission_barrier}. Nucei with energies well below the fission barrier , but above the well in the fission potential, thus sees two thinner barriers, compared to one thick barrier that the nuclei sees if it has energies below this well. They can tunnel through the first barrier, and then exist in an isomer state in the well, before either tunnelling through the second ponential hump, or gamma-decay back to the ground state.  The tunnelling of two thin barriers is more probable than tunnelling through one thick barrier, so the double-humped fission barrier makes more nuclei unstable to spontaneous fission. The existence of the double-humped fission barrier was confirmed when fission isomers were discovered, which are isotopes highly unstable to spontaneous fission \cite{Krane1988} (p.495). 

  \begin{figure}
 	\centering
 	\includegraphics[scale=0.7]{double_humped_fission_barrier.png}
 	\caption{The double humped fission barrier Figure from K.S. Krane p.496 \cite{Krane1988}}
 	\label{fig:smooth_fission_barrier}
 \end{figure}

\subsection{The role of fission in nucleosynthesis}


\section{Method}


\section{Results}

\section{Discussion}

\section{Conclusion}


\bibliographystyle{plain}
\bibliography{TALYS.bib} 






\end{document}
